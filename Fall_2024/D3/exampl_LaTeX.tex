\documentclass{article}\usepackage[]{graphicx}\usepackage[]{xcolor}
% maxwidth is the original width if it is less than linewidth
% otherwise use linewidth (to make sure the graphics do not exceed the margin)
\makeatletter
\def\maxwidth{ %
  \ifdim\Gin@nat@width>\linewidth
    \linewidth
  \else
    \Gin@nat@width
  \fi
}
\makeatother

\definecolor{fgcolor}{rgb}{0.345, 0.345, 0.345}
\newcommand{\hlnum}[1]{\textcolor[rgb]{0.686,0.059,0.569}{#1}}%
\newcommand{\hlstr}[1]{\textcolor[rgb]{0.192,0.494,0.8}{#1}}%
\newcommand{\hlcom}[1]{\textcolor[rgb]{0.678,0.584,0.686}{\textit{#1}}}%
\newcommand{\hlopt}[1]{\textcolor[rgb]{0,0,0}{#1}}%
\newcommand{\hlstd}[1]{\textcolor[rgb]{0.345,0.345,0.345}{#1}}%
\newcommand{\hlkwa}[1]{\textcolor[rgb]{0.161,0.373,0.58}{\textbf{#1}}}%
\newcommand{\hlkwb}[1]{\textcolor[rgb]{0.69,0.353,0.396}{#1}}%
\newcommand{\hlkwc}[1]{\textcolor[rgb]{0.333,0.667,0.333}{#1}}%
\newcommand{\hlkwd}[1]{\textcolor[rgb]{0.737,0.353,0.396}{\textbf{#1}}}%
\let\hlipl\hlkwb

\usepackage{framed}
\makeatletter
\newenvironment{kframe}{%
 \def\at@end@of@kframe{}%
 \ifinner\ifhmode%
  \def\at@end@of@kframe{\end{minipage}}%
  \begin{minipage}{\columnwidth}%
 \fi\fi%
 \def\FrameCommand##1{\hskip\@totalleftmargin \hskip-\fboxsep
 \colorbox{shadecolor}{##1}\hskip-\fboxsep
     % There is no \\@totalrightmargin, so:
     \hskip-\linewidth \hskip-\@totalleftmargin \hskip\columnwidth}%
 \MakeFramed {\advance\hsize-\width
   \@totalleftmargin\z@ \linewidth\hsize
   \@setminipage}}%
 {\par\unskip\endMakeFramed%
 \at@end@of@kframe}
\makeatother

\definecolor{shadecolor}{rgb}{.97, .97, .97}
\definecolor{messagecolor}{rgb}{0, 0, 0}
\definecolor{warningcolor}{rgb}{1, 0, 1}
\definecolor{errorcolor}{rgb}{1, 0, 0}
\newenvironment{knitrout}{}{} % an empty environment to be redefined in TeX

\usepackage{alltt}

\usepackage{amsmath}
\usepackage{geometry}[margin = 1in]
\usepackage{graphicx}
\usepackage{natbib}
 \setcitestyle{citesep={,},aysep={}}
  \bibliographystyle{jwm}

\title{R Bootcamp Fall 2024 LaTeX Example}
\author{Me \and some other dude}
\date{\today}
\IfFileExists{upquote.sty}{\usepackage{upquote}}{}
\begin{document}

\maketitle

\section*{Abstract}
This document is an example. This will turn this script into a one-stop-shop for managing and cleaning data, analyzing those data, generating graphics, and incorporating them into a publication-grade paper.

\noindent Everything else can be inferred from what you see in this script\footnote{We'll point it out with comments along the way.}. However, a final thing to note is any time you make a special environment with a \texttt{\textbackslash begin{}} command, you must close it properly with an \texttt{\textbackslash end{}} command.

\begin{figure}[h]
    \centering
    \includegraphics[width=0.75\textwidth]{figure/meme.jpeg}
    \caption{we love a good meme}
    \label{fig:meme}
\end{figure}

Here is my reference for this cool meme (Figure \ref{fig:meme})


example math mode: 
The equation of a line in slope-intercept form is $y=mx+b$.


MacArther-Levins Niche Overlap is as follows:

\begin{equation}
  \label{eqn:1}
  M_{mf} = \frac{\sum_{i}^{n} p_{im} p_{if}}{\sum p^2_{im}}
\end{equation}

\noindent Everything else...

\newpage
\section*{Introduction}

\section*{Methods}



Items from a code chunk are saved and can be expressed in text using the command \texttt{\textbackslash{}Sexpr\{\}} like so - Results: 2, 9, 16, 16, 23, 30.

% The summary from sum is sum.


\section*{Results}

Some dude name Smith said that Bighorn are dumb \citep{smithEtAl02}.

\section*{Discussion}

\section*{Acknowledgements}


\newpage
\bibliography{exampleBib}


\end{document}
 
